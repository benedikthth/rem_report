% This is samplepaper.tex, a sample chapter demonstrating the
% LLNCS macro package for Springer Computer Science proceedings;
% Version 2.20 of 2017/10/04
%
\documentclass[runningheads]{llncs}

%
\usepackage{graphicx}


\usepackage[sorting=none]{biblatex}

\bibliography{src/citations.bib}{}

% Used for displaying a sample figure. If possible, figure files should
% be included in EPS format.
%
% If you use the hyperref package, please uncomment the following line
% to display URLs in blue roman font according to Springer's eBook style:
% \renewcommand\UrlFont{\color{blue}\rmfamily}

\begin{document}
%
\title{
    Regional differences in attack patterns
    % \thanks{Supported by organization x.}
}
%
%\titlerunning{Abbreviated paper title}
% If the paper title is too long for the running head, you can set
% an abbreviated paper title here
%
\author{Benedikt H. Thordarson }%\and
% Second Author\inst{2,3}\orcidID{1111-2222-3333-4444} \and
% Third Author\inst{3}\orcidID{2222--3333-4444-5555}}
%
% \authorrunning{F. Author et al.}
% First names are abbreviated in the running head.
% If there are more than two authors, 'et al.' is used.
%
\institute{Reykjavik University, Menntavegur 1, Reykjavik, Iceland } % \and
% Springer Heidelberg, Tiergartenstr. 17, 69121 Heidelberg, Germany
% \email{lncs@springer.com}\\
% \url{http://www.springer.com/gp/computer-science/lncs} \and
% ABC Institute, Rupert-Karls-University Heidelberg, Heidelberg, Germany\\
% \email{\{abc,lncs\}@uni-heidelberg.de}}
%
\maketitle              % typeset the header of the contribution
%


\begin{abstract}
    In this paper, I will present the results from
    recording and comparing the attack patterns 
    experienced by honeypot servers in different 
    geographical regions. 

\keywords{Security  \and Honeypot.}
% ToDo: find more keywords

\end{abstract}
%
%
%
\section{Introduction}
\label{sec:introduction}

    If you have a device connected to the
    internet, it is almost certain that some person,
    or an infected machine has probed it, and possibly
    attempted to break into your device. 


    A 2017 study \cite{Britton_Liu}
    used HoneyPot servers to record malicious
    incoming traffic. Over a 24 hour period they 
    experienced 300,103 attempted attacks, or 
    about 3.4 attacks per second on average.
    These attacks were recorded on devices that
    had no associated domain name, and was in no
    way transmitting any data into the internet
    before being probed.

    
    Currently, there is a rising trend of introducing
    the internet-of-things \textit{(IoT)} devices to 
    home, and industrial environments. The number of 
    IoT devices is currently estimated to be 26.66 
    billion, and by 2025, the count is projected to 
    reach 75.44 billion. \cite{statista}

    
    Most of these IoT devices are similar to the honeypots
    described by Britton et al\cite{Britton_Liu}, in that 
    they are usually small devices, not associated with 
    a hostname, and due to the specialized nature, their 
    communication with the wider internet is limited.


    The overproliferation of these IoT devices has given 
    massive fodder to Botnets such as Mirai, which in 2016
    launched a 620 Gbps attack on KrebsOnSecurity.com
    \cite{Brian_Krebs_2016}.

    
    If
\section{Related Work}
\label{sec:relatedWork}


An 2004 analysis by F. Rainal et al. \cite{1324605}
on network traffic, found that from the first 
seconds of being connected to the internet, the 
honeypot is subject to scans. The amount of time 
ago the experiment was conducted makes it difficult
to compare to, since in the 15 years passed since the
publication, the landscape of the internet has changed 
considerably. 
The authors  make no attempt to discern a pattern
in the scans, or attacks made on the server. They also
point out that a sufficiently skilled intruder, may 
realize that he is in a honeypot, and exit so that 
his intentions are not revealed.


In a 2013, I. Koniaris, G. Papadimitriou, and 
P. Nicopolitidis published a paper outlining
their analysis of four months of honeypot 
activity \cite{6624967}.
 While their project was in many ways
similar to this one, their focus was on 
username and password combinations rather than
exploring the correlation of time of day and 
rate of attacks. 


A similar paper from 2011, describes a statistical 
analysis on honeypot activity 
\cite{Song:2011:SAH:1978672.1978676}. 
The authors however do not attempt to discern a 
pattern from the attacks. In addition, the 8 years
since the paper's publication mean that the 
nature of the internet has changed significantly. 


The use of honeypots has led to anti-honeypot technology.
Krawetz \cite{1264861} 
notes in his 2004 article that spammers are re-active,
not proactive, changing their tools only when they 
become ineffective. The same can be claimed with honeypot
technology, as when a detection strategy is discovered
it will be made ineffective. There indeed is certainly 
evidence of an ongoing arms race, as shown by activity
of attackers on the honeypot, when after logging in, the
attackers check automatically if the attacker is inside a 
busybox.


To understand where the traffic to the server originates from
it is important to acknowledge the existence of large-scale 
botnets\cite{AbuRajab:2006:MAU:1177080.1177086}, automatic
probes, and current scanning technology \cite{6657498}.


% \input{src/Sections/Approach.tex}
% \section{Conclusion}
\label{sec:conclusion}

In this paper I have described the process of setting up a HonePot
server with Modern Honey Network. I have also described how I 
processed the data, and how I preformed the statistical analysis.


I found that both the rate of attacks received and the rate 
of attacks launched did not have a strong linear correlation
to the time of day. 


An ANOVA test preformed on the rate of attacks received over 
the time of day showed that there is a difference in rates
significant at the 0.06 level.


If the same test is preformed on the
 time which the attacks
were launched, the difference is now significant at the 0.03
level.


There seems to be a peak of attacks launched at 
05:00 (see figure \ref{fig:over_day}), this may be 
because attackers assume that the security staff is 
not as responsive at such a late hour. But seeing as
how 90\% of attacks come from foreign time zones, this 
seems unlikely to be the reason.


Future work could be repeating this experiment in different
countries, collecting data over longer periods, figuring out
given a pattern, if it can be localized to a timezone.


% \noindent Displayed equations are centered and set on a separate
% line.
% \begin{equation}
% x + y = z
% \end{equation}
% Please try to avoid rasterized images for line-art diagrams and
% schemas. Whenever possible, use vector graphics instead (see
% Fig.~\ref{fig1}).


% \begin{theorem}
% This is a sample theorem. The run-in heading is set in bold, while
% the following text appears in italics. Definitions, lemmas,
% propositions, and corollaries are styled the same way
% \end{theorem}
% 
% the environments 'definition', 'lemma', 'proposition', 'corollary',
% 'remark', and 'example' are defined in the LLNCS documentclass as well.
%
% \begin{proof}
% Proofs, examples, and remarks have the initial word in italics,
% while the following text appears in normal font.
% \end{proof}
    \printbibliography
    % \bibliographystyle{plain}

\end{document}
