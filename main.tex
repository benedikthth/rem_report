% This is samplepaper.tex, a sample chapter demonstrating the
% LLNCS macro package for Springer Computer Science proceedings;
% Version 2.20 of 2017/10/04
%
\documentclass[runningheads]{llncs}

%
\usepackage{graphicx}

\usepackage{todonotes}

% \usepackage[sorting=none]{biblatex}
% \bibliographystyle{ieeetr}
% \bibliography{src/citations.bib}


% Used for displaying a sample figure. If possible, figure files should
% be included in EPS format.
%
% If you use the hyperref package, please uncomment the following line
% to display URLs in blue roman font according to Springer's eBook style:
% \renewcommand\UrlFont{\color{blue}\rmfamily}

\usepackage{float}
\usepackage{bera}% optional: just to have a nice mono-spaced font
\usepackage{listings}
\usepackage{xcolor}

\colorlet{punct}{red!60!black}
\definecolor{background}{HTML}{FAFAFA}
\definecolor{delim}{RGB}{20,105,176}
\colorlet{numb}{magenta!60!black}

\lstdefinelanguage{json}{
    basicstyle=\normalfont\ttfamily,
    numbers=none,
    numberstyle=\scriptsize,
    stepnumber=1,
    numbersep=8pt,
    showstringspaces=false,
    breaklines=false,
    frame=none,% frame=lines,
    backgroundcolor=\color{background},
    literate=
     *{0}{{{\color{numb}0}}}{1}
      {1}{{{\color{numb}1}}}{1}
      {2}{{{\color{numb}2}}}{1}
      {3}{{{\color{numb}3}}}{1}
      {4}{{{\color{numb}4}}}{1}
      {5}{{{\color{numb}5}}}{1}
      {6}{{{\color{numb}6}}}{1}
      {7}{{{\color{numb}7}}}{1}
      {8}{{{\color{numb}8}}}{1}
      {9}{{{\color{numb}9}}}{1}
      {:}{{{\color{punct}{:}}}}{1}
      {,}{{{\color{punct}{,}}}}{1}
      {\{}{{{\color{delim}{\{}}}}{1}
      {\}}{{{\color{delim}{\}}}}}{1}
      {[}{{{\color{delim}{[}}}}{1}
      {]}{{{\color{delim}{]}}}}{1},
}


%%%%%%%%%%%%%%%%%%%%%%%%%%%%%%%%%%%%%%%%%%%
%%%%%%%%%%% Write Nothing Twice %%%%%%%%%%%
%%%%%%%%%%%%%%%%%%%%%%%%%%%%%%%%%%%%%%%%%%%
\newcommand{\serverlocation}{Germany}
\newcommand{\abbrv}[1]{\textit{(#1)}}

\begin{document}
%
\title{
    Analysis of attack patterns on honeypot server
    % \thanks{Supported by organization x.}
}
%
%\titlerunning{Abbreviated paper title}
% If the paper title is too long for the running head, you can set
% an abbreviated paper title here
%
\author{Benedikt H. Thordarson }%\and
% Second Author\inst{2,3}\orcidID{1111-2222-3333-4444} \and
% Third Author\inst{3}\orcidID{2222--3333-4444-5555}}
%
% \authorrunning{F. Author et al.}
% First names are abbreviated in the running head.
% If there are more than two authors, 'et al.' is used.
%
\institute{Reykjavik University, Menntavegur 1, Reykjavik, Iceland } % \and
% Springer Heidelberg, Tiergartenstr. 17, 69121 Heidelberg, Germany
% \email{lncs@springer.com}\\
% \url{http://www.springer.com/gp/computer-science/lncs} \and
% ABC Institute, Rupert-Karls-University Heidelberg, Heidelberg, Germany\\
% \email{\{abc,lncs\}@uni-heidelberg.de}}
%
\maketitle              % typeset the header of the contribution
%


\begin{abstract}
    In this paper I will present the results 
    from analysis of data gathered from over
    560,000 attacks on a honeypot server.
    
\keywords{Security  \and Honeypot.}
% ToDo: find more keywords

\end{abstract}
%
%
%
\section{Introduction}
\label{sec:introduction}

    If you have a device connected to the
    internet, it is almost certain that some person,
    or an infected machine has probed it, and possibly
    attempted to break into your device. 


    A 2017 study \cite{Britton_Liu}
    used HoneyPot servers to record malicious
    incoming traffic. Over a 24 hour period they 
    claim to have experienced 300,103 attempted
    attacks, or about 3.4 attacks per second 
    on average. These attacks were recorded on 
    devices that had no associated domain name, 
    and were in no way transmitting any data 
    into the internet before being probed.

    
    Currently, there is a rising trend of introducing
    the internet-of-things \textit{(IoT)} devices to 
    home, and industrial environments. The number of 
    IoT devices is currently estimated to be 26.66 
    billion, and by 2025, the count is projected to 
    reach 75.44 billion. \cite{statista}

    
    Most of these IoT devices are similar to the honeypots
    described by Britton et al.\cite{Britton_Liu}, in that 
    they are usually small devices, not associated with 
    a hostname, and due to the specialized nature, their 
    communication with the wider internet is limited.


    Due to manufacturer negligence, end-user apathy or 
    a combination of the two, a vast portion of IoT devices
    are left vulnerable.

    The over-proliferation of these IoT devices has given 
    massive fodder to Botnets such as Mirai, which in 2016
    launched a 620 Gbps attack on KrebsOnSecurity.com
    \cite{Brian_Krebs_2016}.


    To combat the attempts of automated infectious devices
    and novice would-be attackers, many measures have been
    generated, such as Intrusion-Detection-Systems(IDS)
    \cite{4693640} and, Intrusion-Prevention-Systems(IPS) like 
    firewalls. A system that sits somewhere in between the
    two, are HoneyPot servers. Devices designed to pose as
    vulnerable hosts and fool would-be attackers. 
    HoneyPots can be designed to respond to 
    different attacks, such as dictionary attacks on SSH servers,
    Telnet, or even wordpress plugin vulnerabilities.

    
    Observing a pattern of attack rate over the day can lead
    to a number of interesting hypotheses, but I will leave any
    conjecture until after the data has been collected and 
    adequately analyzed.

    
    
\section{Related Work}
\label{sec:relatedWork}
\section{Methods}
  \label{sec:methods}
  

  To gather the data for this report, I will use the Modern 
  Honey Network \abbrv{MHN} to manage 
  and deploy the HoneyPot server. MHN will collect data about
  the attacks as they take place, and store it in a database.


  For hosting, I choose DigitalOcean, for the ability to
  cheaply run my server in a foreign country.
  

  
  
  In this study, I focused on SSH brute force attacks.
  To collect data I will use the Cowrie\cite{Cowrie}
  honeypot. An medium interaction SSH and Telnet honeypot
  designed to log brute force attacks and the shell 
  interaction performed by the attacker. A sample session 
  log can be seen in listing \ref{lst:data}

  
\begin{figure}[h!]
\begin{lstlisting}[language=json, 
    caption={Sample attack data gathered from the London HoneyPot. (ID fields omitted for brevity.)},
    captionpos=b, label={lst:data}]

    { 
        "payload": {
            "commands": [], 
            "credentials": [], 
            "endTime": "2019-02-26T14:43:38.820991Z", 
            "hashes": [], 
            "hostIP": "68.183.213.102", 
            "hostPort": 22, 
            "loggedin": [
            "root", 
            "admin"
            ], 
            "peerIP": "5.188.86.174", 
            "peerPort": 52790, 
            "protocol": "ssh", 
            "startTime": "2019-02-26T14:40:38.669266Z", 
            "ttylog": null, 
            "unknownCommands": [], 
            "urls": [], 
            "version": "'SSH-2.0-OpenSSH_7.3'"
        }, 
        "timestamp": "2019-02-26T14:43:38.840000"
        }, 

    \end{lstlisting}
\end{figure}

  While MHN has a rest API, it is not very useful,
  instead the data had to be manually fetched from
  the MHN server's database.

  This data includes a timestamp, IP-address,
  target-protocol,
  the credentials that were used to log into the 
  server, and even the commands that were run when
  the attacker thought he had gained access.
  
  
  By using publicly available databases like
  MaxMind's GeoIP2 City and Country CSV Databases
  \cite{maxminds}, the IP addresses can be translated
  into countries of origins.
  

  It is important to note, that while each attack is 
  timestamped, the timestamp is in the timezone of the 
  attacked server, not the attacker. To more accurately
  analyse the rate of launched attacks over day, the
  geo-location can also be used to adjust the timestamp
  to represent the time of which the attack was launched.
  It bears acknowledging that there are methods available
  to fool geolocation, such as using VPN, TOR, and for
  operations of large enough scale, IP address space hijacking.
  

  
  The handling of data in the ways as described above
  can be achieved by using python, and the existing Pandas
  \cite{pandas} library, which makes the handling and
  aggregation of data easy, and also allows for plotting
  the data.
  
  
\section{Results}
\label{sec:results}

The recorded results are diverse, and incumbent with 
information. The honeypot records both time stamps, and 
interaction logs. Analysis of each will be handled separately 
in subsections \ref{sec:time_analysis} and 
\ref{sec:session_analysis}. 

It is worth noting that at this point in the project,
this section, along with the rest of the writing should
be considered to be work-in-progress.

\subsection{Statistical analysis of attack data}
\label{sec:time_analysis}

    The honeypot has been active since 12. of January
    2019, and until the 1. of March has logged 
    562.276 individual ssh attacks. 

    It was surprising that over 65 percent of attacks 
    originated in Ireland. A single IP address is the
    origin of 11.5 percent of attacks from Ireland 
    (see figure \ref{fig:ireland_breakdown})
    
    \begin{figure}
        \includegraphics[width=0.8\textwidth]{src/images/ireland_breakdown.png}
        \caption{Attacks from Ireland broken down by source address}
        \label{fig:ireland_breakdown}
    \end{figure}

    A further investigation of the top-offender, 
    \texttt{5.188.86.174} reveals that this host
    is a tor exit node or relay \cite{tor}, which allows users
    to anonymize their internet activity. A nmap scan reveals
    a large number of services running on the host. As of \today
    there has been no attempt made to contact the host to notify
    them that their devices are complicit in potentially 
    illegal activity.


    If the rate of attempted attacks over time is plotted, 
    a pattern seems to start to emerge. (See figure \ref{fig:over_monday}
    for a plot of a typical monday)
    \begin{figure}
        \includegraphics[width=0.8\textwidth]{src/images/OverMonday.png}
        \label{fig:over_monday}
    \end{figure}

    If the same done over the week (see figure \ref{fig:over_week}), there is a seemingly
    non-random pattern that can be discerned, in that 
    over the weekend, the rate becomes more varied.

    \begin{figure}
        \includegraphics[width=0.8\textwidth]{src/images/week_unfiltered.png} 
        \label{fig:over_week}
    \end{figure}

    However, before any statistical analysis on the data is
    preformed, no conclusion should be drawn from apparent
    patterns.

\subsection{Session log analysis}
\label{sec:session_analysis}

    The cowrie honeypot also captures the actions of 
    the attacker, as well as the files which he attempts
    to download and run on what he thinks is a compromised
    server. 


    One of the more intriguing factors that a number of the
    attack logs show, is an attempt to access a command
    called \texttt{'\\gisdfoewrsfdf'}. A command that 
    has I do not know what does, and a search yields no
    definitive results to what this command or script is. 

    Amongst the other things that are attempted, 
    are wiping the \texttt{./ssh/authorized\_keys} file and
    inserting a key, presumably of another infected
    device. 

    Twice during the period of 12. January to 1. March, 
    did an attacker try to download, and start an IRC bot.

    
\section{Discussion}
\label{sec:discussion}

\subsection{Prevalence of Irish attacks}
\label{sec:irish_analysis}

Referring to section \ref{sec:origin_analysis}, it 
came as a rather large surprise that Ireland was the most 
common source of attackers. 

\begin{figure}[H]
    \centering
    \includegraphics[width=0.8\textwidth]{src/images/ip_breakdown.png}
    \caption{Total attacks broken down by source address}
    \label{fig:attacks_by_ip}
\end{figure}


In fact, looking into the top offenders, 


\texttt{'5.188.86.174'},
\texttt{'5.188.86.211'},
\texttt{'5.188.87.53'},
\texttt{'5.188.87.54'}, and 
\texttt{'5.188.87.55'} 

reveals that all of them 
are owned by \texttt{'Petersburg 
Internet Network ltd.'}(PIN) which has been
implicated in both route-hijacking \cite{bogus_routing} 
and allegedly with a gang of cyber criminals \cite{petersburg}

% \section{Results}
\label{sec:results}

The recorded results are diverse, and incumbent with 
information. The honeypot records both time stamps, and 
interaction logs. Analysis of each will be handled separately 
in subsections \ref{sec:time_analysis} and 
\ref{sec:session_analysis}. 

It is worth noting that at this point in the project,
this section, along with the rest of the writing should
be considered to be work-in-progress.

\subsection{Statistical analysis of attack data}
\label{sec:time_analysis}

    The honeypot has been active since 12. of January
    2019, and until the 1. of March has logged 
    562.276 individual ssh attacks. 

    It was surprising that over 65 percent of attacks 
    originated in Ireland. A single IP address is the
    origin of 11.5 percent of attacks from Ireland 
    (see figure \ref{fig:ireland_breakdown})
    
    \begin{figure}
        \includegraphics[width=0.8\textwidth]{src/images/ireland_breakdown.png}
        \caption{Attacks from Ireland broken down by source address}
        \label{fig:ireland_breakdown}
    \end{figure}

    A further investigation of the top-offender, 
    \texttt{5.188.86.174} reveals that this host
    is a tor exit node or relay \cite{tor}, which allows users
    to anonymize their internet activity. A nmap scan reveals
    a large number of services running on the host. As of \today
    there has been no attempt made to contact the host to notify
    them that their devices are complicit in potentially 
    illegal activity.


    If the rate of attempted attacks over time is plotted, 
    a pattern seems to start to emerge. (See figure \ref{fig:over_monday}
    for a plot of a typical monday)
    \begin{figure}
        \includegraphics[width=0.8\textwidth]{src/images/OverMonday.png}
        \label{fig:over_monday}
    \end{figure}

    If the same done over the week (see figure \ref{fig:over_week}), there is a seemingly
    non-random pattern that can be discerned, in that 
    over the weekend, the rate becomes more varied.

    \begin{figure}
        \includegraphics[width=0.8\textwidth]{src/images/week_unfiltered.png} 
        \label{fig:over_week}
    \end{figure}

    However, before any statistical analysis on the data is
    preformed, no conclusion should be drawn from apparent
    patterns.

\subsection{Session log analysis}
\label{sec:session_analysis}

    The cowrie honeypot also captures the actions of 
    the attacker, as well as the files which he attempts
    to download and run on what he thinks is a compromised
    server. 


    One of the more intriguing factors that a number of the
    attack logs show, is an attempt to access a command
    called \texttt{'\\gisdfoewrsfdf'}. A command that 
    has I do not know what does, and a search yields no
    definitive results to what this command or script is. 

    Amongst the other things that are attempted, 
    are wiping the \texttt{./ssh/authorized\_keys} file and
    inserting a key, presumably of another infected
    device. 

    Twice during the period of 12. January to 1. March, 
    did an attacker try to download, and start an IRC bot.

    
\section{Conclusion}
\label{sec:conclusion}

In this paper I have described the process of setting up a HonePot
server with Modern Honey Network. I have also described how I 
processed the data, and how I preformed the statistical analysis.


I found that both the rate of attacks received and the rate 
of attacks launched did not have a strong linear correlation
to the time of day. 


An ANOVA test preformed on the rate of attacks received over 
the time of day showed that there is a difference in rates
significant at the 0.06 level.


If the same test is preformed on the
 time which the attacks
were launched, the difference is now significant at the 0.03
level.


There seems to be a peak of attacks launched at 
05:00 (see figure \ref{fig:over_day}), this may be 
because attackers assume that the security staff is 
not as responsive at such a late hour. But seeing as
how 90\% of attacks come from foreign time zones, this 
seems unlikely to be the reason.


Future work could be repeating this experiment in different
countries, collecting data over longer periods, figuring out
given a pattern, if it can be localized to a timezone.


% \noindent Displayed equations are centered and set on a separate
% line.
% \begin{equation}
% x + y = z
% \end{equation}
% Please try to avoid rasterized images for line-art diagrams and
% schemas. Whenever possible, use vector graphics instead (see
% Fig.~\ref{fig1}).


% \begin{theorem}
% This is a sample theorem. The run-in heading is set in bold, while
% the following text appears in italics. Definitions, lemmas,
% propositions, and corollaries are styled the same way
% \end{theorem}
% 
% the environments 'definition', 'lemma', 'proposition', 'corollary',
% 'remark', and 'example' are defined in the LLNCS documentclass as well.
%
% \begin{proof}
% Proofs, examples, and remarks have the initial word in italics,
% while the following text appears in normal font.
% \end{proof}
    % \printbibliography
    \clearpage
    \bibliographystyle{ieeetr}
    \bibliography{src/Citations}
    % \bibliographystyle{plain}

\end{document}
