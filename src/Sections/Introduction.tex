
\begin{abstract}
    In this paper I will present the results 
    from analysis of data gathered from over
    812.013 attacks on a honeypot server.
    

% \keywords{Security  \and Honeypot.}
% ToDo: find more keywords

\end{abstract}
%
%
%
\section{Introduction}
\label{sec:introduction}

    If you have a device connected to the
    internet, it is almost certain that some person,
    or an infected machine has probed it, and possibly
    attempted to break into your system. 


    To combat the attempts of automated infectious devices
    and novice would-be attackers, a number of techniques 
    have been developed, such as Intrusion-Detection-Systems \abbrv{IDS}
    \cite{4693640} and, Intrusion-Prevention-Systems \abbrv{IPS} like 
    firewalls. HoneyPot servers sit somewhere in between the
    two. HoneyPots are devices designed to pose as
    vulnerable hosts and fool would-be attackers. 
    HoneyPots can be designed to respond to 
    different attacks, such as dictionary attacks on SSH servers,
    Telnet, or even WordPress plugin vulnerabilities.


    A 2017 study \cite{Britton_Liu}
    used HoneyPot servers to record malicious
    incoming traffic. Over a 24 hour period they 
    claim to have experienced 300,103 attempted
    attacks, or about 3.4 attacks per second 
    on average. These attacks were recorded on 
    devices that had no associated domain name, 
    and were in no way transmitting any data 
    into the internet before being probed.

    
    Currently, there is a rising trend of introducing
    the internet-of-things \abbrv{IoT} devices to 
    home, and industrial environments. The number of 
    IoT devices in 2019 is estimated to be 26.66 
    billion, and by 2025, the count is projected to 
    reach 75.44 billion. \cite{statista}

    
    Most of these IoT devices are similar to the honeypots
    described by Britton et al.\cite{Britton_Liu}, in that 
    they are usually small devices, not associated with 
    a hostname, and due to the specialized nature, their 
    communication with the wider internet is limited.


    Due to manufacturer negligence, end-user ignorance or 
    a combination of the two, a vast portion of IoT devices
    are left vulnerable. It is worth pointing out that these
    devices are sold as general consumer devices, thus do
    not assume any technical knowledge of the end-user.


    The over-proliferation of these IoT devices has 
    provided a large number of vulnerable machines 
    for Botnets such as Mirai, which in 2016
    launched a 620 Gbps attack on KrebsOnSecurity.com
    \cite{Brian_Krebs_2016}.


    The problem of vulnerable devices is not new, 
    even before the rise of IoT, 
    there were still large numbers of vulnerable machines 
    on the internet.


    
    This paper describes a study on attack rates on SSH 
    honeypots.
    Section \ref{sec:relatedWork} introduces parallel work 
        and relevant literature.
    Section \ref{sec:methods} discusses the way that the 
        data is collected, processed, and analyzed.
    Section \ref{sec:results} will detail the analysis of
        the data, both of the time-series analysis of 
        attack rates, and also of the attacker behavior.
    The paper concludes, and presents potential future
    work in section \ref{sec:conclusion}. 