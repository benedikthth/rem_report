
\begin{abstract}
    In this paper I will present the results 
    from analysis of data gathered from over
    560,000 attacks on a honeypot server.
    
\keywords{Security  \and Honeypot.}
% ToDo: find more keywords

\end{abstract}
%
%
%
\section{Introduction}
\label{sec:introduction}

    If you have a device connected to the
    internet, it is almost certain that some person,
    or an infected machine has probed it, and possibly
    attempted to break into your device. 


    A 2017 study \cite{Britton_Liu}
    used HoneyPot servers to record malicious
    incoming traffic. Over a 24 hour period they 
    claim to have experienced 300,103 attempted
    attacks, or about 3.4 attacks per second 
    on average. These attacks were recorded on 
    devices that had no associated domain name, 
    and were in no way transmitting any data 
    into the internet before being probed.

    
    Currently, there is a rising trend of introducing
    the internet-of-things \textit{(IoT)} devices to 
    home, and industrial environments. The number of 
    IoT devices is currently estimated to be 26.66 
    billion, and by 2025, the count is projected to 
    reach 75.44 billion. \cite{statista}

    
    Most of these IoT devices are similar to the honeypots
    described by Britton et al.\cite{Britton_Liu}, in that 
    they are usually small devices, not associated with 
    a hostname, and due to the specialized nature, their 
    communication with the wider internet is limited.


    Due to manufacturer negligence, end-user apathy or 
    a combination of the two, a vast portion of IoT devices
    are left vulnerable.

    The over-proliferation of these IoT devices has given 
    massive fodder to Botnets such as Mirai, which in 2016
    launched a 620 Gbps attack on KrebsOnSecurity.com
    \cite{Brian_Krebs_2016}.


    To combat the attempts of automated infectious devices
    and novice would-be attackers, many measures have been
    generated, such as Intrusion-Detection-Systems(IDS)
    \cite{4693640} and, Intrusion-Prevention-Systems(IPS) like 
    firewalls. A system that sits somewhere in between the
    two, are HoneyPot servers. Devices designed to pose as
    vulnerable hosts and fool would-be attackers. 
    HoneyPots can be designed to respond to 
    different attacks, such as dictionary attacks on SSH servers,
    Telnet, or even wordpress plugin vulnerabilities.

    
    Observing a pattern of attack rate over the day can lead
    to a number of interesting hypotheses, but I will leave any
    conjecture until after the data has been collected and 
    adequately analyzed.

    
    