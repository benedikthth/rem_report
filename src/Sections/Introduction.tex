
\begin{abstract}
    In this paper, I will present the results from
    recording and comparing the attack patterns 
    experienced by honeypot servers in different 
    geographical regions. 

\keywords{Security  \and Honeypot.}
% ToDo: find more keywords

\end{abstract}
%
%
%
\section{Introduction}
\label{sec:introduction}

    If you have a device connected to the
    internet, it is almost certain that some person,
    or an infected machine has probed it, and possibly
    attempted to break into your device. 


    A 2017 study \cite{Britton_Liu}
    used HoneyPot servers to record malicious
    incoming traffic. Over a 24 hour period they 
    experienced 300,103 attempted attacks, or 
    about 3.4 attacks per second on average.
    These attacks were recorded on devices that
    had no associated domain name, and was in no
    way transmitting any data into the internet
    before being probed.

    
    Currently, there is a rising trend of introducing
    the internet-of-things \textit{(IoT)} devices to 
    home, and industrial environments. The number of 
    IoT devices is currently estimated to be 26.66 
    billion, and by 2025, the count is projected to 
    reach 75.44 billion. \cite{statista}

    
    Most of these IoT devices are similar to the honeypots
    described by Britton et al\cite{Britton_Liu}, in that 
    they are usually small devices, not associated with 
    a hostname, and due to the specialized nature, their 
    communication with the wider internet is limited.


    The overproliferation of these IoT devices has given 
    massive fodder to Botnets such as Mirai, which in 2016
    launched a 620 Gbps attack on KrebsOnSecurity.com
    \cite{Brian_Krebs_2016}.

    
    If