\section{Methods}
  \label{sec:methods}
  

  To gather the data for this report, I will use the Modern 
  Honey Network \abbrv{MHN} to manage 
  and deploy the HoneyPot server. MHN will collect data about
  the attacks as they take place, and store it in a database.


  For hosting, I choose DigitalOcean, for the ability to
  cheaply run my server in a foreign country.
  

  
  
  In this study, I focused on SSH brute force attacks.
  To collect data I will use the Cowrie\cite{Cowrie}
  honeypot. An medium interaction SSH and Telnet honeypot
  designed to log brute force attacks and the shell 
  interaction performed by the attacker. A sample session 
  log can be seen in listing \ref{lst:data}

  
\begin{figure}[H]
\begin{lstlisting}[language=json, 
    caption={Sample attack data gathered from the London HoneyPot. (ID fields omitted for brevity.)},
    captionpos=b, label={lst:data}]

    { 
        "payload": {
            "commands": [], 
            "credentials": [], 
            "endTime": "2019-02-26T14:43:38.820991Z", 
            "hashes": [], 
            "hostIP": "68.183.213.102", 
            "hostPort": 22, 
            "loggedin": [
            "root", 
            "admin"
            ], 
            "peerIP": "5.188.86.174", 
            "peerPort": 52790, 
            "protocol": "ssh", 
            "startTime": "2019-02-26T14:40:38.669266Z", 
            "ttylog": null, 
            "unknownCommands": [], 
            "urls": [], 
            "version": "'SSH-2.0-OpenSSH_7.3'"
        }, 
        "timestamp": "2019-02-26T14:43:38.840000"
    }, 

    \end{lstlisting}
\end{figure}

  While MHN has a rest API, it is not very useful,
  instead the data had to be manually fetched from
  the MHN server's database.

  This data includes a timestamp, IP-address,
  target-protocol,
  the credentials that were used to log into the 
  server, and even the commands that were run when
  the attacker thought he had gained access.
  
  
  By using publicly available databases like
  MaxMind's GeoIP2 City and Country CSV Databases
  \cite{maxminds}, the IP addresses can be translated
  into countries of origins.
  

  It is important to note, that while each attack is 
  timestamped, the timestamp is in the timezone of the 
  attacked server, not the attacker. To more accurately
  analyse the rate of launched attacks over day, the
  geo-location can also be used to adjust the timestamp
  to represent the time of which the attack was launched.
  It bears acknowledging that there are methods available
  to fool geolocation, such as using VPN, TOR, and for
  operations of large enough scale, IP address space hijacking.
  

  
  The handling of data in the ways as described above
  can be achieved by using python, and the existing Pandas
  \cite{pandas} library, which makes the handling and
  aggregation of data easy, and also allows for plotting
  the data.
  
  