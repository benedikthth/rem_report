\section{Related Work}
\label{sec:relatedWork}


An 2004 analysis by F. Rainal et al. \cite{1324605}
on network traffic, found that from the first 
seconds of being connected to the internet, the 
honeypot is subject to scans. The amount of time 
ago the experiment was conducted makes it difficult
to compare to, since in the 15 years passed since the
publication, the landscape of the internet has changed 
considerably. 
The authors  make no attempt to discern a pattern
in the scans, or attacks made on the server. They also
point out that a sufficiently skilled intruder, may 
realize that he is in a honeypot, and exit so that 
his intentions are not revealed.


In a 2013, I. Koniaris, G. Papadimitriou, and 
P. Nicopolitidis published a paper outlining
their analysis of four months of honeypot 
activity \cite{6624967}.
 While their project was in many ways
similar to this one, their focus was on 
username and password combinations rather than
exploring the correlation of time of day and 
rate of attacks. 


A similar paper from 2011, describes a statistical 
analysis on honeypot activity 
\cite{Song:2011:SAH:1978672.1978676}. 
The authors however do not attempt to discern a 
pattern from the attacks. In addition, the 8 years
since the paper's publication mean that the 
nature of the internet has changed significantly. 


The use of honeypots has led to anti-honeypot technology.
Krawetz \cite{1264861} 
notes in his 2004 article that spammers are re-active,
not proactive, changing their tools only when they 
become ineffective. The same can be claimed with honeypot
technology, as when a detection strategy is discovered
it will be made ineffective. There indeed is certainly 
evidence of an ongoing arms race, as shown by activity
of attackers on the honeypot, when after logging in, the
attackers check automatically if the attacker is inside a 
busybox.


To understand where the traffic to the server originates from
it is important to acknowledge the existence of large-scale 
botnets\cite{AbuRajab:2006:MAU:1177080.1177086}, automatic
probes, and current scanning technology \cite{6657498}.

