\section{Related Work}
\label{sec:relatedWork}


There has not been a whole lot of work 
put into attack pattern analysis of honeypot
servers.


An 2004 analysis by F. Rainal et al. \cite{1324605}
on network traffic, found that from the first 
seconds of being connected to the internet, the 
honeypot is subject to scans. The amount of time 
ago the experiment was conducted makes it difficult
to compare to, since in the 15 years passed since the
publication, the landscape of the internet has changed 
considerably. 
The authors  make no attempt to discern a pattern
in the scans, or attacks made on the server. They also
point out that a sufficiently skilled intruder, may 
realize that he is in a honeypot, and exit so that 
his intentions are not revealed.



The use of honeypots has led to anti-honeypot technology.
Krawetz \cite{1264861} 
notes in his 2004 article that spammers are re-active,
not proactive, changing their tools only when they 
become ineffective. The same can be claimed with honeypot
technology, as when a detection strategy is discovered
it will be made ineffective. The two sides will be forever
caught in a evolutionary arms race.



To understand where the traffic to the server originates from
it is important to acknowledge the existence of large-scale 
botnets\cite{AbuRajab:2006:MAU:1177080.1177086}, automatic
probes, and current scanning technology \cite{6657498}.

